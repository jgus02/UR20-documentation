%% It is just an empty TeX file.
%% Write your code here.
Historically, interactions between humans and industrial robots have been major safety concerns. Early factory environments did not focus on human safety in these conditions. This often resulted in strict separation between workers and machines, with physical barriers and lock-out protocols put in place to prevent accident or injury. These set-ups created restricted zones around machinery, and created hazardous conditions whenever human interaction or intervention with the equipment was necessary. Human intervention with a palletizing robot is often necessary for troubleshooting errors in the production line, such as a misplaced or trapped box on the conveyor belt. In these situations, workers must enter the robot’s operating zone to correct the issue, which is dangerous in traditional set-ups.

In a factory palletization setting, the UR20 Cobot offers a safety advantage compared to traditional heavy machinery. Whereas conventional automated systems which often require physical barriers or guarded zones to protect workers from potential harm, the UR20 is specifically engineered for safe human interaction. Its integrated sensors detect physical touch and respond to any notable resistance by stopping movement immediately. This built-in responsiveness drastically reduces the risk of accidents, even when the cobot is performing tasks that involve lifting heavy payloads. This makes it no longer necessary to guard the operating area with restricted zones or physical barriers, allowing for more efficient operation and greater worker safety.

Our implementation of the UR20 Cobot includes proximity detection and camera systems capable of recognizing human presence. These technologies allow the cobot to anticipate and respond to the presence of nearby workers, reducing the possibility of workplace accidents.

