This proposal will be focusing on the UR20 Cobot, although the original design was for the Mitsubishi RV8 arm.

\subsection{Current \& Pending Support}
This project's funding currently consists of only the default \$800 funding from the CSE department. More funding may be provided with sufficient justification, but no other funding sources are currently known.

    Current total funds:
    \begin{itemize}
        \item \$800 Default Budget
    \end{itemize}

\subsection{Final Budget}
The final cost of our project as of April 2025 is listed here.
The UR20 Cobot has a provided safety scanner, wiring box, and conveyor belt. A gripper will need to be purchased or designed in order for it to perform a palletizing application.

The item that ended up being chosen was a custom-design suction-cup gripper using parts borrowed from the Senior Design labs and 3D printed materials. \$300 worth of suction cups were purchased before an unused, higher-quality alternative was found in the labs.  


This table records the cost of all acquired items for the project.
\begin{table}[H]
\centering
    \caption{Budget itemization}
    \begin{tabular}{|l|l|}
        \hline
        \textbf{Item} & \textbf{Price} \\ \hline
        End effector and components & \$300+ \\ \hline
        Vacuum pump  and components & \$0 \\ \hline
        Mounting bracket & \$0\\ \hline
        Feeder belt & \$0 \\ \hline
        Photoeye detector & \$30 \\ \hline
        Miscellaneous small parts (screws, shaft collars, etc) & \$100 \\ \hline
        Raspberry Pi and camera & \$0 \\ \hline
        Boxes and pallet & \$0 \\ \hline
    \end{tabular}
\end{table}

\subsection{Preliminary Budget}
 The original options considered at the creation of this document are listed here.

\begin{itemize}
    \item A conveyor belt will likely be necessary for simulation of a real-world factory palletizing environment's margin of error.
    \item A vacuum generator will be needed, either included in the gripper or acquired separately.
    \item For palletizing, it is likely that a suction grip will be the most effective. However, those grips can be extremely expensive, likely coming in at a minimum of \$2000 unless used parts can be found. This is due to the necessity of a vacuum generator attached before the actual gripper in order to provide it power.
    \item Some more light-weight grippers have been designed for other robots such as MyCobot, and may be translatable to our robot. This would cost only \$400, and should be explored by reaching out to those manufacturers for questions.
    \item We could also, as the most time-consuming option, take inspiration from existing designs of vacuum pumps or end effectors and design one of the components, using the remaining money to invest in the other component. 
\end{itemize}


This table assumes we will purchase the cheapest versions of the above options instead of making them ourselves.
\begin{table}[H]
\centering
    \caption{Budget planning}
    \begin{tabular}{|l|l|}
        \hline
        \textbf{Item} & \textbf{Price} \\ \hline
        End effector and components & \$400+ \\ \hline
        Vacuum pump  and components & \$200+ \\ \hline
        Mounting bracket & \$100\\ \hline
        Feeder belt & \$100 \\ \hline
    \end{tabular}
\end{table}
