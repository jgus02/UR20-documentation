%% It is just an empty TeX file.
%% Write your code here.
The UR20 Palletizing Cobot project is located in ERB 335, a designated laboratory area. A rectangular area approximately 10x5 feet in dimension is necessary for palletizing. For safety reasons, the robot will not operate if any obstacle, especially a person, is within this radius. The lab area includes the appropriate electrical infrastructure accommodating the power requirements of the UR20 robot. It will be necessary for the robot to be bolted into the ground for safe operation. This will be handled by the lab coordinator.

Equipment needed includes 3D printers (already present in the lab), a conveyor belt (already present in the lab), an air compressor (already present in the lab). Materials needed to conduct testing will be a pallet (borrowed from UTA) and boxes of consistent dimensions (printer paper boxes borrowed from UTA). 

Components will be 3D printed or purchased using the team budget. Circuits will be made by team members using materials borrowed from UTA Senior Design labs. If time and budget permits, some components will be finalized as aluminum parts, as opposed to 3D printed material. Some equipment will be provided by team members, such as a Raspberry Pi used for the human shape recognition camera stretch goal.