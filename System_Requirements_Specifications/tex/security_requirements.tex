This section defines the security standards that will be taken to avoid malicious tampering involving the hardware or software of the Cobot. In the event that the Cobot's security is confirmed to be compromised, continued use will be halted until approval is given by the course instructor.

\subsection{Physical Security}
\subsubsection{Description}
The Cobot, including the programming tools and control systems, shall be located in an area that only approved persons may access. In the event that an unapproved person gains access to the Cobot, the possibility of tampering should be ruled out for both the hardware and software of the Cobot.
\subsubsection{Source}
LJCJ Team Decision
\subsubsection{Constraints}
The Senior Design lab is frequently used by a large number of students. As it would be impossible work in the case that all of those students were considered a safety concern, these studenst shall be considered approved persons.
\subsubsection{Standards}
N/A
\subsubsection{Priority}
High

\subsection{Cyber Security}
\subsubsection{Description}
In the event that network access is necessary, the Cobot should not be connected to insecure or unknown networks. When connection is no longer needed, the Cobot should be disconnected from the network.

Similarly, the Cobot should not be connected to insecure or unknown devices.

Removeable media shall only be used when its contents are completely known and approved of by at least one member of the team. Do not insert removeable media of dubious origin into the Cobot. 
\subsubsection{Source}
LJCJ Team Decision
\subsubsection{Constraints}
The Cobot may need to be connected to UTA Wi-Fi and our team's computers, which are not very secure. However, it is unlikely that the course instructor will fund alternatives, so these must be used regardless. 
\subsubsection{Standards}
N/A
\subsubsection{Priority}
High